% This is samplepaper.tex, a sample chapter demonstrating the
% LLNCS macro package for Springer Computer Science proceedings;
% Version 2.20 of 2017/10/04
%
\documentclass[runningheads]{llncs}
%
\usepackage{graphicx}
\usepackage{hyperref}
% Used for displaying a sample figure. If possible, figure files should
% be included in EPS format.
%
% If you use the hyperref package, please uncomment the following line
% to display URLs in blue roman font according to Springer's eBook style:
% \renewcommand\UrlFont{\color{blue}\rmfamily}

\begin{document}
%
\title{Communicating with Executives.}
%
%\titlerunning{Abbreviated paper title}
% If the paper title is too long for the running head, you can set
% an abbreviated paper title here
%
\author{Namra Solanki, ID: 40232377}
%
% \authorrunning{F. Author et al.}
% First names are abbreviated in the running head.
% If there are more than two authors, 'et al.' is used.
%
\institute{Concordia University, Montreal QC, Canada \\
% Springer Heidelberg, Tiergartenstr. 17, 69121 Heidelberg, Germany
% \email{lncs@springer.com}\\
% \url{http://www.springer.com/gp/computer-science/lncs} \and
% ABC Institute, Rupert-Karls-University Heidelberg, Heidelberg, Germany\\
\email{namra.solanki@mail.concordia.ca}}
%
\maketitle              % typeset the header of the contribution
%
\begin{abstract}
A review article explores the evolution of communication strategies from an engineering manager to a CEO, shedding light on challenges faced when interacting with upper management. Early frustrations stemmed from perceived disconnect between non-technical aspects of the business and technical initiatives. With organizational growth, inherent complexity demands strategic communication. \\

The transition to a CEO role unveiled a two-fold challenge: competing demands for attention and the need for more effective communication. Tips for executive communication and presentations are distilled for graduate-level consideration. The granularity of conversations should align with company size, emphasizing independence in managing tactical concerns while communicating broader themes. Proactive communication is crucial, urging managers to push updates even in non-crisis situations. In larger companies, bursty updates serve as valuable reminders. A three-step communication process with executives is outlined, emphasizing clarity and a structured approach. \\

Writing a narrative before presentations aids in identifying gaps and considering diverse perspectives. The article advises against evasiveness, preparing for unexpected questions, and utilizing data for a results-oriented approach. The importance of leading meetings with a clear goal is highlighted, respecting executive time and favoring clarity over ambiguity. These strategies collectively form a comprehensive guide for effective executive communication and leadership amidst organizational growth.

\keywords{Communication  \and Project Management \and Leadership \and Data-driven Decision Making \and Effective Presentation}
\end{abstract}
%
%
%
\section{Table of Contents}

\begin{table}
\centering
\caption{Table of Contents}\label{tab1}
\begin{tabular}{|l|c|}
\hline
Title &  Page Number\\
\hline
Abstract & 1\\
Introduction &  1\\
Background Material & 1\\
Methods and Methodology & 1\\
Results Obtained & 1\\
Conclusion and Future Work & 1\\
References & 1\\
\hline
\end{tabular}
\end{table}

\subsubsection{Sample Heading (Third Level)} Only two levels of
headings should be numbered. Lower level headings remain unnumbered;
they are formatted as run-in headings.

\paragraph{Sample Heading (Fourth Level)}
The contribution should contain no more than four levels of
headings. Table~\ref{tab1} gives a summary of all heading levels.

\begin{table}
\caption{Table captions should be placed above the
tables.}\label{tab1}
\begin{tabular}{|l|l|l|}
\hline
Heading level &  Example & Font size and style\\
\hline
Title (centered) &  {\Large\bfseries Lecture Notes} & 14 point, bold\\
1st-level heading &  {\large\bfseries 1 Introduction} & 12 point, bold\\
2nd-level heading & {\bfseries 2.1 Printing Area} & 10 point, bold\\
3rd-level heading & {\bfseries Run-in Heading in Bold.} Text follows & 10 point, bold\\
4th-level heading & {\itshape Lowest Level Heading.} Text follows & 10 point, italic\\
\hline
\end{tabular}
\end{table}


\noindent Displayed equations are centered and set on a separate
line.
\begin{equation}
x + y = z
\end{equation}
Please try to avoid rasterized images for line-art diagrams and
schemas. Whenever possible, use vector graphics instead (see
Fig.~\ref{fig1}).

\begin{figure}
\includegraphics[width=\textwidth]{fig1.eps}
\caption{A figure caption is always placed below the illustration.
Please note that short captions are centered, while long ones are
justified by the macro package automatically.} \label{fig1}
\end{figure}

\begin{theorem}
This is a sample theorem. The run-in heading is set in bold, while
the following text appears in italics. Definitions, lemmas,
propositions, and corollaries are styled the same way.
\end{theorem}
%
% the environments 'definition', 'lemma', 'proposition', 'corollary',
% 'remark', and 'example' are defined in the LLNCS documentclass as well.
%
\begin{proof}
Proofs, examples, and remarks have the initial word in italics,
while the following text appears in normal font.
\end{proof}
For citations of references, we prefer the use of square brackets
and consecutive numbers. Citations using labels or the author/year
convention are also acceptable. The following bibliography provides
a sample reference list with entries for journal
articles~\cite{ref_article1}, an LNCS chapter~\cite{ref_lncs1}, a
book~\cite{ref_book1}, proceedings without editors~\cite{ref_proc1},
and a homepage~\cite{ref_url1}. Multiple citations are grouped
\cite{ref_article1,ref_lncs1,ref_book1},
\cite{ref_article1,ref_book1,ref_proc1,ref_url1}.
%
% ---- Bibliography ----
%
% BibTeX users should specify bibliography style 'splncs04'.
% References will then be sorted and formatted in the correct style.
%
% \bibliographystyle{splncs04}
% \bibliography{mybibliography}
%
\begin{thebibliography}{8}
\bibitem{ref_article1}
Title: Motivation and Stakeholder Acceptance in Technology-driven Change Management: Implications for the Engineering Manager. \href{https://www.tandfonline.com/doi/abs/10.1080/10429247.2008.11431764}{Paper Link}, Author: \href{https://scholar.google.com/citations?user=7_LPDDQAAAAJ&hl=en}{Suzanna Long}, \href{https://emse.mst.edu/facultystafffacilities/emsefaculty/davidspurlock/}{David G. Spurlock}

\bibitem{ref_article2}
Title: A Transactional Model of Communication. \href{https://www.taylorfrancis.com/chapters/edit/10.4324/9781315080918-5/transactional-model-communication-dean-barnlund}{Book Link}, Author: \href{https://scholar.google.com/citations?user=7_LPDDQAAAAJ&hl=en}{Suzanna Long}

\bibitem{ref_article3}
Title: Project Management Communication: a Systems Approach \href{https://www.researchgate.net/publication/220195825_Project_Management_Communication_a_Systems_Approach}{Paper Link}, Author: Sharlett Gillard, Jane Johansen

\bibitem{ref_article4}
Authon: Loo, R. (2002). Title: Journaling: A Learning Tool for Project Management Training and Team-building. Project Management Journal. \href{https://doi.org/10.1177/875697280203300407}{DOI link}

\bibitem{ref_article5}
Author: \href{https://scholar.google.com/citations?user=o355sjwAAAAJ&hl=en}{Rodriguez, Pedro}. (2017). Conceptual model of communication theories within project process. INNOVA Research Journal. 2. 42-51. 10.33890/innova.v2.n3.2017.131. 

\bibitem{ref_article6}
Authors: Smit, Marius and Bond-Barnard, Taryn and Steyn, H. Fabris-Rotelli, Inger. (2017). Title: Email communication in project management: A bane or a blessing?. South African Journal of Information Management. in press. 10.4102/sajim.v19i1.826.\href{https://www.researchgate.net/publication/313770452_Email_communication_in_project_management_A_bane_or_a_blessing} {Paper Link}

\bibitem{ref_article7}
Authors: \href{https://www.researchgate.net/profile/Jonas-Stier}{Stier, Jonas} and \href{https://www.researchgate.net/profile/Margareta-Sandstroem}{Sandström, Margareta. (2009).} Communicative challenges in multinational project work: Obstacles and tools for reaching common understandings. Journal of Intercultural Communication. 9. 10.36923/jicc.v9i3.489. \href{https://www.researchgate.net/publication/285593311_Communicative_challenges_in_multinational_project_work_Obstacles_and_tools_for_reaching_common_understandings}{Paper Link}

\bibitem{ref_article8}
Author: \href{https://www.scopus.com/authid/detail.uri?authorId=6602223274}{John Wateridge},
Title: Training for IS/IT project managers: A way forward, International Journal of Project Management, ISSN 0263-7863, \href{https://doi.org/10.1016/S0263-7863(96)00085-3}{Paper Link}.

\bibitem{ref_article9}
Author: \href{https://scholar.google.com/citations?user=WjhruLQAAAAJ&hl=en}{BG Zulch},
Title: Communication: The Foundation of Project Management, Procedia Technology, ISSN 2212-0173,
https://doi.org/10.1016/j.protcy.2014.10.054. \href{https://www.sciencedirect.com/science/article/pii/S2212017314002813}{Paper Link}

\bibitem{ref_article10}
Author: \href{https://scholar.google.com/citations?user=WjhruLQAAAAJ&hl=en}{BG Zulch},
Title: Leadership Communication in Project Management, Procedia - Social and Behavioral Sciences, ISSN 1877-0428, https://doi.org/10.1016/j.sbspro.2014.03.021.
\href{https://www.sciencedirect.com/science/article/pii/S1877042814021120}{Paper Link}
\end{thebibliography}
\end{document}
